\documentclass[a4paper,11pt]{article}

\setlength{\textwidth}{15.0cm}
\setlength{\textheight}{24.0cm}
\setlength{\topmargin}{0cm}
\setlength{\headsep}{0cm}
\setlength{\headheight}{0cm}
\pagestyle{plain}

\usepackage[utf8]{inputenc}
\usepackage{amsmath, amsfonts, mathtools, amsthm, amssymb}
\usepackage{import}
\usepackage{pdfpages}
\usepackage{transparent}
\usepackage{xcolor}

\title{25. Moments}

\author{}
\date{}


\begin{document}
\maketitle

We define the $g$-moment of a random variable and discuss finiteness of the $g$-moment of $X_{t}$ for a Lévy process
$\left\{X_{t}\right\}$. \\

DEFINITION 25.1. Let $g(x)$ be a nonnegative measurable function on $\mathbb{R}^{d}$. We call
$\int g(x) \mu(\mathrm{d} x)$ the $g$-moment of a measure $\mu$ on $\mathbb{R}^{d}$. We call $E[g(X)]$ the $g$-moment of
a random variable $X$ on $\mathbb{R}^{d}$. \\

DEFINITION 25.2. A function $g(x)$ on $\mathbb{R}^{d}$ is called submultiplicative if it is nonnegative
and there is a constant $a>0$ such that

\begin{equation*}
    g(x+y) \leq a g(x) g(y) \text { for } x, y \in \mathbb{R}^{d} \tag{25.1}
\end{equation*}

\noindent A function bounded on every compact set is called locally bounded. \\

THEOREM 25.3 ($g$-Moment). Let $g$ be a submultiplicative, locally bounded, measurable
function on $\mathbb{R}^{d}$. Then, finiteness of the $g$-moment is not a time dependent
distributional property in the class of Lévy processes. Let $\left\{X_{t}\right\}$ be a Lévy process
on $\mathbb{R}^{d}$ with Lévy measure $\nu$. Then, $X_{t}$ has finite $g$-moment for every $t>0$ if
and only if $[\nu]_{\{|x|>1\}}$ has finite $g$-moment. \\

The following facts indicate the wide applicability of the theorem. \\

PROPOSITION 25.4. \\
(i) The product of two submultiplicative functions is submultiplicative. \\
(ii) If $g(x)$ is submultiplicative on $\mathbb{R}^{d}$, then so is $g(c x+\gamma)^{\alpha}$
with $c \in \mathbb{R}$, $\gamma \in \mathbb{R}^{d}$, and $\alpha>0$. \\
(iii) Let $0<\beta \leq 1$. Then the following functions are submultiplicative:

$$
    \begin{gathered}
        |x| \vee 1,\left|x_{j}\right| \vee 1, x_{j} \vee 1, \exp \left(|x|^{\beta}\right), \exp \left(\left|x_{j}\right|^{\beta}\right) \\
        \exp \left(\left(x_{j} \vee 0\right)^{\beta}\right), \log (|x| \vee e), \log \left(\left|x_{j}\right| \vee e\right), \log \left(x_{j} \vee \mathrm{e}\right) \\
        \log \log \left(|x| \vee e^{e}\right), \log \log \left(\left|x_{j}\right| \vee e^{e}\right), \log \log \left(x_{j} \vee e^{e}\right)
    \end{gathered}
$$

Here $x_{j}$ is the $j$ th component of $x$.

\begin{proof}

    (i) Immediate from the definition. \\
    (ii) Let $g_{1}(x)=g(c x), g_{2}(x)=g(x+\gamma)$, and $g_{3}(x)=g(x)^{\alpha}$.
    Then it follows from (25.1) that $g_{1}(x+y) \leq a g_{1}(x) g_{1}(y), g_{2}(x+y) \leq
        a^{2} g(-\gamma) g_{2}(x) g_{2}(y)$, and $g_{3}(x+y) \leq a^{\alpha} g_{3}(x) g_{3}(y)$. \\
    (iii) Let $h(u)$ be a positive increasing function on $\mathbb{R}$ such that, for some
    $b \geq 0, h(u)$ is flat on $(-\infty, b]$ and $\log h(u)$ is concave on $[b, \infty)$. Then
    $h(u)$ is submultiplicative on $\mathbb{R}$. In fact, for $u, v \geq b$, the function $f(u)=\log h(u)$ satisfies

    $$
        \begin{aligned}
             & f(u+b)-f(u) \leq f(2 b)-f(b) \\
             & f(u+v)-f(v) \leq f(u+b)-f(b)
        \end{aligned}
    $$

    and hence

    $$
        f(u+v) \leq f(u+b)-f(b)+f(v) \leq f(2 b)-2 f(b)+f(u)+f(v)
    $$

    which shows


    \begin{equation*}
        h(u+v) \leq \text { const } h(u) h(v) \tag{25.2}
    \end{equation*}


    It follows that (25.2) holds for all $u, v \in \mathbb{R}$. The functions
    $u \vee 1, \exp \left((u \vee 0)^{\beta}\right)$, $\log (u \vee \mathrm{e})$, and
    $\log \log \left(u \vee \mathrm{e}^{\mathrm{e}}\right)$ fulfil the conditions on
    $h(u)$. By (25.2) and by the increasingness of $h$, the functions $h(|x|), h\left(\left|x_{j}\right|\right)$, and
    $h\left(x_{j}\right)$ are submultiplicative on $\mathbb{R}^{d}$.

\end{proof}

We prove Theorem 25.3 after three lemmas.

LEMMA 25.5. If $g(x)$ is submultiplicative and locally bounded, then

\begin{equation*}
    g(x) \leq b e^{c|x|} \tag{25.3}
\end{equation*}
with some $b>0$ and $c>0$.

\begin{proof}
    Choose $b$ in such a way that $\sup _{|x| \leq 1} g(x) \leq b$ and $a b>1$. If $n-1<|x| \leq n$,
    then
    $$
        g(x) \leq a^{n-1} g\left(\frac{1}{n} x\right)^{n} \leq a^{n-1} b^{n} \leq b(a b)^{|x|}
    $$
    which shows (25.3).
\end{proof}

LEMMA 25.6. Let $\mu$ be an infinitely divisible distribution on $\mathbb{R}$ with Lévy measure $\nu$
supported on a bounded set. Then $\widehat{\mu}(z)$ can be extended to an entire function on $\mathbb{C}$.

\begin{proof}
    There is a finite $a>0$ such that $S_{\nu} \subset[-a, a]$. The Lévy representation of $\widehat{\mu}(z)$ is written as
    $$
        \widehat{\mu}(z)=\exp \left[-\frac{1}{2} A z^{2}+\int_{[-a, a]}\left(\mathrm{e}^{\mathrm{i} z x}-1-\mathrm{i} z x\right) \nu(\mathrm{d} x)+\mathrm{i} \gamma^{\prime} z\right]
    $$
    with some $\gamma^{\prime} \in \mathbb{R}$. The right-hand side is meaningful even if $z$ is complex. Denote
    this function by $\Phi(z)$. Then $\Phi(z)$ is an entire function, since we can exchange the order of
    integration and differentiation.
\end{proof}


LEMMA 25.7. If $\mu$ is a probability measure on $\mathbb{R}$ and $\widehat{\mu}(z)$ is extendible to an entire
function on $\mathbb{C}$, then $\mu$ has finite exponential moments, that is, it has finite $e^{c|x|}$-moment for every $c>0$.

\begin{proof}
    It follows from Proposition 2.5(x) that $\alpha_{n}=\int x^{n} \mu(\mathrm{d} x)$ and
    $\beta_{n}=\int|x|^{n} \mu(\mathrm{d} x)$ are finite for any $n \geq 1$. Since
    $\frac{\mathrm{d}^{n} \widehat{\mu}}{\mathrm{d} z^{n}}(0)=\mathrm{i}^{n} \alpha_{n}$, we have

    $$
        \widehat{\mu}(z)=\sum_{n=0}^{\infty} \frac{1}{n !} \mathrm{i}^{n} \alpha_{n} z^{n}
    $$

    the radius of convergence of the right-hand side being infinite. Notice that $\beta_{2 k}=\alpha_{2 k}$ and
    $\beta_{2 k+1} \leq \frac{1}{2}\left(\alpha_{2 k+2}+\alpha_{2 k}\right)$, since
    $|x|^{2 k+1} \leq \frac{1}{2}\left(x^{2 k+2}+x^{2 k}\right)$. It follows that

    $$
        \int \mathrm{e}^{|x|} \mu(\mathrm{d} x)=\sum_{n=0}^{\infty} \frac{1}{n !} \beta_{n} c^{n}<\infty
    $$
    completing the proof.

\end{proof}

\begin{proof}[Proof of Theorem 25.3]

    Let $\nu_{0}=[\nu]_{\{|x| \leq 1\}}$ and $\nu_{1}=[\nu]_{\{|x|>1\}}$. Construct
    independent Lévy processes $\left\{X_{t}^{0}\right\}$ and $\left\{X_{t}^{1}\right\}$ on $\mathbb{R}^{d}$ such
    that $\left\{X_{t}\right\} =_{d} \left\{X_{t}^{0}+ X_{t}^{1}\right\}$ and
    $\left\{X_{t}^{1}\right\}$ is compound Poisson with Lévy measure $\nu_{1}$. Let $\mu_{0}$ and $\mu_{1}$ be
    the distributions of $X_{1}^{0}$ and $X_{1}^{1}$, respectively.

    Suppose that $X_{t}$ has finite $g$-moment for some $t>0$. It follows from

    $$
        E\left[g\left(X_{t}\right)\right]=\iint g(x+y) \mu_{0}^{t}(dx) \mu_{1}^{t}(dy)
    $$

    that $\int g(x+y) \mu_{1}{ }^{t}(dy)<\infty$ for some $x$. This means

    $$
        \sum_{n=0}^{\infty} \frac{t^{n}}{n !} \int g(x+y) \nu_{1}^{n}(dy)<\infty
    $$

    Since $g(y) \leq a g(-x) g(x+y) \leq a b e^{c|x|} g(x+y)$ by Lemma 25.5, we get


    \begin{equation*}
        \sum_{n=0}^{\infty} \frac{t^{n}}{n !} \int g(y) \nu_{1}^{n}(dy)<\infty \tag{25.4}
    \end{equation*}

    Hence $\int g(y) \nu_{1}(d y)<\infty$.

    Conversely, suppose that $\int g(y) \nu_{1}(dy)<\infty$. Let us prove
    that $E\left[g\left(X_{t}\right)\right]$ $<\infty$ for every $t$. By the submultiplicativity,

    $$
        \begin{aligned}
            \int g(y) \nu_{1}^{n}(dy) & =\int \cdots \int g\left(y_{1}+\cdots+y_{n}\right) \nu_{1}\left(dy_{1}\right) \ldots \nu_{1}\left(d y_{n}\right) \\
                                      & \leq a^{n-1}\left(\int g(y) \nu_{1}(dy)\right)^{n}
        \end{aligned}
    $$

    Hence we have (25.4) for every $t$. That is, $X_{t}^{1}$ has finite $g$-moment. Since

    $$
        E\left[g\left(X_{t}\right)\right] \leq a b E\left[\mathrm{e}^{\mathrm{c}\left|X_{t}^{0}\right|}\right] E\left[g\left(X_{t}^{1}\right)\right]
    $$

    by (25.1) and (25.3), it remains only to show that $E\left[e^{\mathrm{ec} X_{i}^{0}}\right]<\infty$.
    Let $X_{j}^{0}(t)$, $1 \leq j \leq d$, be the components of $X_{t}^{0}$. Then

    $$
        E\left[e^{c\left|X_{i}^{0}\right|}\right] \leq E\left[\exp \left(c \sum_{j=1}^{d}\left|X_{j}^{0}(t)\right|\right)\right] \leq E\left[\prod_{j=1}^{d}\left(e^{c X_{j}^{0}(t)}+e^{-c X_{j}^{0}(t)}\right)\right]
    $$

    which is written as a sum of a finite number of terms of the form $E\left[\exp X_{t}^{\sharp}\right]$
    with $X_{t}^{\sharp}$ being a linear combination of $X_{j}^{0}(t), 1 \leq j \leq d$. Since $\left\{X_{t}^{\sharp}\right\}$
    is a Lévy process on $\mathbb{R}$ with Lévy measure supported on a bounded set (use Proposition 11.10),
    $E\left[\exp X_{t}^{\sharp}\right]$ is finite by virtue of Lemmas 25.6 and 25.7. This proves all statements in the theorem.

\end{proof}

Corollary 25.8. Let $\alpha>0,0<\beta \leq 1$, and $\gamma \geq 0$. None of the properties
$\int|x|^{\alpha} \mu(\mathrm{d} x)<\infty, \int(0 \vee \log |x|)^{\alpha} \mu(\mathrm{d} x)<\infty$, and
$\int|x|^{\gamma} e^{\alpha|x|^{\beta}} \mu(\mathrm{d} x)<$ $\infty$ is time dependent in the class
of Lévy processes. For a Lévy process on $\mathbb{R}^{d}$ with Lévy measure $\nu$, each of the properties
is expressed by the corresponding property of $[\nu]_{\{|x|>1\}}$.

\begin{proof}
    This follows from Theorem 25.3 and Proposition 25.4.
\end{proof}

REMARK 25.9. There is a nonnegative measurable function $g(x)$ satisfying (25.3) such that finiteness of the $g$-moment is a
time dependent distributional property in the class of Lévy processes. For example, let $g(x)=\left(1 \wedge|x|^{-\alpha}\right)
    \mathrm{e}^{|x|}$ with $\alpha>0$. Consider a $\Gamma$-process $\left\{X_{t}\right\}$ with $E[X_{1}]=1$. Then it is easy to see
that $E\left[g\left(X_{t}\right)\right]<\infty$ if and only if $t<\alpha$. This process has
$\nu=x^{-1} \mathrm{e}^{-x} 1_{(0, \infty)}(x) \mathrm{d} x$, so that $\nu_{1}$ has finite $g$-moment (Example 8.10). \\

EXAMPLE 25.10. Let $\left\{X_{t}\right\}$ be a non-trivial semi-stable process on $\mathbb{R}^{d}$ with index
$\alpha \in(0,2)$. Then, for every $t>0, E\left[\left|X_{t}\right|^{\eta}\right]$ is finite or infinite
according as $0<\eta<\alpha$ or $\eta \geq \alpha$, respectively. To see this, notice that the argument
in the proofs of Theorem 13.15 and Proposition 14.5 gives

$$
    \int_{S_{n}(b)}|x|^{\eta} \nu(\mathrm{d} x)=b^{n(\eta-\alpha)} \int_{S_{0}(b)}|x|^{\eta} \nu(\mathrm{d} x)
$$

and hence $\int_{|x|>1}|x|^{\eta} \nu(\mathrm{d} x)<\infty$ if and only if $\eta<\alpha$; apply
Corollary 25.8. In particular, for a stable process on $\mathbb{R}$ with parameters
$(\alpha, \beta, \tau, c)$ (Definition 14.16), $E\left[X_{t}\right]=\tau t$ if
$1<\alpha<2$ (use Proposition 2.5(ix)). The following explicit results are known.
If $0<\alpha<1$ and $\left\{X_{t}\right\}$ is a stable subordinator with
$E\left[\mathrm{e}^{-u X_{t}}\right]=\mathrm{e}^{-t \epsilon^{\prime} u^{\alpha}}$ (Example 24.12),
then, for $-\infty<\eta<\alpha$,

\begin{equation*}
    E\left[X_{t}^{\eta}\right]=\left(t c^{\prime}\right)^{\eta / \alpha} \frac{\Gamma\left(1-\frac{\eta}{\alpha}\right)}{\Gamma(1-\eta)} \tag{25.5}
\end{equation*}


which is shown by Wolfe [508] and Shanbhag and Sreehari [419] (Exercise 29.17).
If $0<\alpha<2$ and $\left\{X_{t}\right\}$ is symmetric and $\alpha$-stable on $\mathbb{R}$
with $E\left[e^{\mathbf{i} X X_{t}}\right]=e^{-t c|z|^{\alpha}}$ (Theorem 14.14), then, for $-1<\eta<\alpha$,

\begin{equation*}
    E\left[\left|X_{t}\right|^{\eta}\right]=(t c)^{\eta / \alpha} \frac{2^{\eta} \Gamma\left(\frac{1+\eta}{2}\right) \Gamma\left(1-\frac{\eta}{\alpha}\right)}{\sqrt{\pi} \Gamma\left(1-\frac{\eta}{2}\right)} \tag{25.6}
\end{equation*}

as is shown in [419]. \\

EXAMPLE 25.11. If $\left\{X_{t}\right\}$ is a Lévy process on $\mathbb{R}$ with Lévy
measure supported on $(-\infty, 0]$, then $E\left[\mathrm{e}^{c X_{t}}\right]<\infty$
for every $c>0$ and $t>0$. Use Theorem 25.3 for $g(x)=e^{c x}$. For instance, a stable
process on $\mathbb{R}$ with $1 \leq \alpha<2$ and $\beta=-1$ satisfies this assumption
although it has support $\mathbb{R}$ for every $t>0$ (Theorem 24.10(i)). \\

EXAMPLE 25.12. Let $\left\{X_{t}\right\}$ be a Lévy process on $\mathbb{R}^{d}$ generated
by $(A, \nu$, 1). In components, $X_{t}=\left(X_{j}(t)\right), \gamma=\left(\gamma_{j}\right)$,
and $A=\left(A_{j k}\right)$. Then $X_{t}$ has finite mean for $t>0$
if and only if $\int_{\{|x| >1\}} |x| \nu(\mathrm{d} x)<\infty$.
When this condition is met, we can find $m_{j}(t)=E\left[X_{j}(t)\right]$ expressed as


\begin{equation*}
    m_{j}(t)=t\left(\int_{|x|>1} x_{j} \nu(\mathrm{d} x)+\gamma_{j}\right)=t \gamma_{1, j}, \quad j=1, \ldots, d \tag{25.7}
\end{equation*}

differentiating $\widehat{\mu}(z)$ (Proposition 2.5(ix)). Here $\gamma_{1, j}$ is the $j$ th
component of the center $\gamma_{1}$ in (8.8). Similarly, $E\left[\left|X_{t}\right|^{2}\right]<\infty$
for all $t>0$ if and only if $\int_{|x|>1}|x|^{2} \nu(\mathrm{d} x)<\infty$. In this case,

$$
    v_{j k}(t)=E\left[\left(X_{j}(t)-m_{j}(t)\right)\left(X_{k}(t)-m_{k}(t)\right)\right], \quad j, k=1, \ldots, d
$$

the $(j, k)$ elements of the covariance matrix of $X(t)$, are expressed as


\begin{equation*}
    v_{j k}(t)=t\left(A_{j k}+\int_{\mathbb{R}^{d}} x_{j} x_{k} \nu(\mathrm{d} x)\right) \tag{25:8}
\end{equation*}


Theorem 25.3 shows that, for a Lévy process $\left\{X_{t}\right\}$ with Lévy measure $\nu$,
the tails of $P_{X_{t}}$ and $\nu$ have a kind of similarity. Are they actually equivalent
in some class of Lévy processes? This question was answered by Embrecht, Goldie, and Veraverbeke [109]
for subordinators. We state their result without proof in two remarks below. \\

Definition 25.13. A probability measure $\mu$ on $[0, \infty)$ is called
subexponential if $\mu(x, \infty)>0$ for every $x$ and

\begin{equation*}
    \lim _{x \rightarrow \infty} \frac{\mu^{n}(x, \infty)}{\mu(x, \infty)}=n \quad \text { for } n=2,3, \ldots \tag{25.9}
\end{equation*}

The class of probability measures satisfying (25.9) above was introduced by Chistyakov [65].
The condition can be weakened. Specifically, if

$$
    \lim \sup _{x \rightarrow \infty} \frac{\mu^{2}(x, \infty)}{\mu(x, \infty)} \leq 2
$$

then $\mu$ is subexponential. The meaning of (25.9) is as follows. Let $\left\{Z_{j}\right\}$
be independent nonnegative random variables each with distribution $\mu$
and let $S_{n}=$ $\sum_{j=1}^{n} Z_{j}$ and $M_{n}=\max _{1 \leq j \leq n} Z_{j}$.
Then $\mu$ satisfies (25.9) if and only if

$$
    P\left[S_{n}>x\right] \sim P\left[M_{n}>x\right], \quad x \rightarrow \infty, \quad \text { for } n=2,3, \ldots
$$

In fact, $P\left[S_{n}>x\right]=\mu^{n}(x, \infty)$ and

$$
    \begin{aligned}
        P\left[M_{n}>x\right] & =\sum_{j=1}^{n} P\left[X_{1} \leq x, \ldots, X_{j-1} \leq x, X_{j}>x\right]                                  \\
                              & =\sum_{j=1}^{n} \mu(0, x)^{j-1} \mu(x, \infty) \sim n \mu(x, \infty) \quad \text { as } x \rightarrow \infty
    \end{aligned}
$$

REMARK 25.14. A basic result on subexponentiality is as follows. If $\left\{X_{t}\right\}$
be a subordinator with Lévy measure $\nu$, then the following conditions are equivalent [109]:

(1) $\nu(1, \infty)>0$ and $\frac{1}{\nu(1, \infty)}[\nu]_{(1, \infty)}$ is subexponential; \\

(2) $P_{X_{t}}$ is subexponential for every $t>0$; \\

(3) $P_{X_{t}}$ is subexponential for some $t>0$; \\

(4) $P\left[X_{t}>x\right] \sim t \nu(x, \infty), x \rightarrow \infty$, for every $t>0$; \\

(5) $P\left[X_{t}>x\right] \sim t \nu(x, \infty), x \rightarrow \infty$, for some $t>0$. \\

Some of the consequences of subexponentiality are as follows. Let $\mu$ be a subexponential probability measure on $[0, \infty)$.
Then,\\

(1) for any $y \in \mathbb{R}, \mu(x-y, \infty) / \mu(x, \infty) \rightarrow 1$ as $x \rightarrow \infty$; \\

(2) for every $\varepsilon>0, \int_{[0, \infty)} \mathrm{e}^{\varepsilon x} \mu(\mathrm{d} x)=\infty$; \\

(3) if $\mu^{\prime}$ is a probability measure on $[0, \infty)$ satisfying $\lim _{x \rightarrow \infty}
    \frac{\mu^{\prime}(x, \infty)}{\mu(x, \infty)}=c$ for some $c \in(0, \infty)$, then $\mu^{\prime}$ is subexponential. \\

A function $L(x)$ is called slowly varying at $\infty$ if $L(x) \neq 0$ and $L(c x) \sim L(x)$, $x \rightarrow \infty$,
for any $c>0$. A function $f(x)$ is called regularly varying of index $\eta$ at $\infty$ if $f(x)=x^{\eta} L(x)$ with $L(x)$
slowly varying at $\infty$. \\

REMARK 25.15. A sufficient condition for subexponentiality is as follows. If $\mu$ is a probability
measure on $[0, \infty)$ such that $\mu(x, \infty)$ is regularly varying of index $-\alpha$ at $\infty$
with some $\alpha \geq 0$, then $\mu$ is subexponential. In the case of an infinitely divisible distribution
with Lévy measure $\nu$, we can also apply this to $\frac{1}{\nu(1, \infty)}[\nu]_{(1, \infty)}$. For example,
the Pareto distribution (Remark 8.12) and onesided stable distributions (by the form of the Lévy measures in Remark 14.4)
are subexponential. As examples not covered by this sufficient condition, the Weibull distribution with parameter $0<\alpha<1$
and the log-normal distribution in Remark 8.12 are subexponential.

For related results and references on subexponentiality, see the recent book [110] of Embrecht, Klüppelberg, and Mikosch.\\

REMARK 25.16. Gribel [156] extends a part of the assertions in Remark 25.14 as follows. Let $h(x)$ be a
nonnegative continuous function on $[0, \infty)$ de creasing to 0 as $x \rightarrow \infty$ such that

\begin{equation*}
    -\int_{0}^{x} h(x-y) \mathrm{d} h(y)=O(h(x)), \quad x \rightarrow \infty \tag{25.10}
\end{equation*}

Let $\mu$ be an infinitely divisible distribution on $\mathbb{R}$ and let $\nu$ be its Lévy measure.
Then the following hold as $x \rightarrow \infty: \mu(x, \infty)=O(h(x))$ if and only
if $\nu(x, \infty)=$ $O(h(x)) ; \mu(x, \infty)=o(h(x))$ if and only if $\nu(x, \infty)=o(h(x))$.
Examples of functions $h(x)$ satisfying the conditions above are $h(x)=(1+x)^{-\alpha}(1+\log (1+$ $x))^{-\beta}$
with $\alpha>0, \beta \geq 0$ or with $\alpha=0, \beta>0$, and $h(x)=\mathrm{e}^{-c x^{\alpha}}$ with $c>0$, $0<\alpha<1$.
A sufficient condition for (25.10) is that $\sup _{x} \frac{h(x)}{h(2 x)}<\infty$.

When $g(x)=\mathrm{e}^{\langle c, x\rangle}$, the $g$-moment of a Lévy process is explicitly expressible.
We define, for $w=\left(w_{j}\right)_{1 \leq j \leq d}$ and $v=\left(v_{j}\right)_{1 \leq j \leq d}$ in $\mathbb{C}^{d}$,
the inner product $\langle w, v\rangle=\sum_{j=1}^{d} w_{j} v_{j}$ (not the Hermitian
inner product $\sum_{j=1}^{d} w_{j} \overline{v_{j}}$ ).
We write $\operatorname{Re} w=\left(\operatorname{Re} w_{j}\right)_{1 \leq j \leq d} \in \mathbb{R}^{d}$. Let $D=\left\{x \in \mathbb{R}^{d}:|x| \leq 1\right\}$. \\

THEOREM 25.17 (Exponential moment). Let $\left\{X_{t}\right\}$ be a Lévy process on $\mathbb{R}^{d}$
generated by $(A, \nu, \gamma)$. Let

$$
    C=\left\{c \in \mathbb{R}^{d}: \int_{|x|>1} \mathrm{e}^{\langle c, x\rangle} \nu(\mathrm{d} x)<\infty\right\}
$$

(i) The set $C$ is convex and contains the origin. \\

(ii) $c \in C$ if and only if $E \mathrm{e}^{\left\langle c, X_{t}\right\rangle}<\infty$ for some $t>0$ or,
equivalently, for every $t>0$.\\

(iii) If $w \in \mathbb{C}^{d}$ is such that $\operatorname{Re} w \in C$, then


\begin{equation*}
    \Psi(w)=\frac{1}{2}\langle w, A w\rangle+\int_{\mathbb{R}^{d}}\left(\mathrm{e}^{\langle w, x\rangle}-1-\langle w, x\rangle 1_{D}(x)\right) \nu(\mathrm{d} x)+\langle\gamma, w\rangle \tag{25.11}
\end{equation*}


is definable, $E\left|\mathrm{e}^{\left(w, X_{t}\right)}\right|<\infty$, and


\begin{equation*}
    E\left[\mathrm{e}^{\left\langle\boldsymbol{w}, X_{t}\right\rangle}\right]=\mathrm{e}^{t \Psi(w)} \tag{25.12}
\end{equation*}

\begin{proof}
    (i) Obviously $C$ contains the origin. If $c_{1}$ and $c_{2}$ are in $C$, then, for any $0<r<1$ and $s=1-r$,

    $$
        \int_{|x|>1} e^{\left\langle r c_{1}+s c_{2}, x\right\rangle} \nu(\mathrm{d} x) \leq\left(\int_{|x|>1} e^{\left\langle c_{1}, x\right\rangle } \nu(\mathrm{d} x)\right)^{\tau}\left(\int_{|x|>1} e^{\left\langle c_{2}, x\right\rangle} \nu(\mathrm{d} x)\right)^{s}<\infty
    $$

    by Hölder's inequality. Hence $C$ is convex.\\

    (ii) The function $g(x)=\mathrm{e}^{(c, x)}$ is clearly submultiplicative. Hence Theorem 25.3 gives the assertion. \\

    (iii). Any linear transformation $U$ of $\mathbb{R}^{d}$ to $\mathbb{R}^{d}$ can be uniquely extended to a linear transformation of $\mathbb{C}^{d}$ to $\mathbb{C}^{d}$. Regarding $U$ as a $d \times d$ matrix, it is easy to see that $\langle w, U v\rangle=\left\langle U^{\prime} w, v\right\rangle$ for $w, v \in \mathbb{C}^{d}$, where $U^{\prime}$ is the transpose of $U$. Now let $\operatorname{Re} w \in C$. Then $\int_{|x|>1}\left|\mathrm{e}^{\langle\boldsymbol{w}, x\rangle}\right| \nu(\mathrm{d} x)=\int_{|x|>1} \mathrm{e}^{\langle\operatorname{Re} w, x\rangle} \nu(\mathrm{d} x)<\infty$, which shows that $\Psi(w)$ of (25.11) is definable and finite. Also, $E\left|\mathrm{e}^{\left\langle w, X_{t}\right\rangle}\right|=$ $E e^{\left(R e v, X_{t}\right\rangle}<\infty$ by (ii). Let us show (25.12) in three steps.

    Step 1. Let $e_{1}$ be the unit vector with first component 1. Assume that $e_{1} \in C$. Let us prove (25.12) for all $w=\left(w_{j}\right)_{1 \leq j \leq d}$ with $\operatorname{Re} w_{1} \in[0,1]$ and\\
    $\operatorname{Re} w_{j}=0,2 \leq j \leq d$. Fix $t>0$ and $w_{2}, \ldots, w_{d} \in \mathbb{C}$ with $\operatorname{Re} w_{j}=0$, $2 \leq j \leq d$, and regard $w_{1}$ as variable in $F=\left\{w_{1} \in \mathbb{C}: \operatorname{Re} w_{1} \in[0,1]\right\}$. Consider $f\left(w_{1}\right)=E \mathrm{e}^{\left\langle\boldsymbol{w}, X_{t}\right\rangle}$.
    Then $f\left(w_{1}\right)$ is continuous on $F$, since

    $$
        \left|e^{\langle\boldsymbol{w}, X(t)\rangle}\right|=e^{\left(\operatorname{Re} w_{1}\right) X_{1}(t)} \leq\left(\operatorname{Re} w_{1}\right) e^{X_{1}(t)}+\left(1-\operatorname{Re} w_{1}\right) \leq e^{X_{1}(t)}+1
    $$

    by the convexity of $\mathrm{e}^{u X_{1}(t)}$ in $u$, where $X_{1}(t)$ is the first component of $X(t)$.
    Moreover, $f\left(w_{1}\right)$ is analytic in the interior of $F$, since it is the limit of the analytic
    functions $E\left[e^{\left\langle\boldsymbol{v}, X_{t}\right\rangle} ;\left|X_{t}\right| \leq n\right]$ as $n \rightarrow \infty$.
    Similarly, $h\left(w_{1}\right)=$ $\mathrm{e}^{\mathrm{tw}(w)}$ is continuous on $F$ and analytic in the interior of $F$.
    If $\operatorname{Re} w_{1}=0$, then $f\left(w_{1}\right)=h\left(w_{1}\right)$, which is the Lévy-Khintchine representation
    of $P_{X_{i}}$. Therefore, as in the proof of Theorem 24.11, the principle of reflection and the uniqueness theorem yield (25.12)
    when $\operatorname{Re} w_{1} \in[0,1]$.

    Step 2. Let $U$ be a linear transformation from $\mathbb{R}^{d}$ onto $\mathbb{R}^{d}$. Let $Y_{t}=$ $U X_{t}$.
    Then $\left\{Y_{t}\right\}$ is a Lévy process with generating triplet $\left(A_{U}, \nu_{U}, \gamma_{U}\right)$
    by Proposition 11.10. Write

    $$
        C_{U}=\left\{c \in \mathbb{R}^{d}: \int_{|x|>1} e^{\langle c, x\rangle} \nu_{U}(\mathrm{~d} x)<\infty\right\}
    $$

    Since $\nu_{U}=\nu U^{-1}$, we have $C_{U}=\left(U^{\prime}\right)^{-1} C$. Given $w \in \mathbb{C}^{d}$
    satisfying Re $w \in$ $C$, let $v=\left(U^{\prime}\right)^{-1} w$. Then $\operatorname{Re} v \in C_{U}$. Define

    $$
        \Psi_{U}(v)=\frac{1}{2}\left\langle v, A_{U} v\right\rangle+\int_{\mathbb{R}^{d}}\left(\mathrm{e}^{\langle v, x\rangle}-1-\langle v, x\rangle 1_{D}(x)\right) \nu_{U}(\mathrm{~d} x)+\left\langle\gamma_{U}, v\right\rangle
    $$

    We claim that if

    \begin{equation*}
        E\left[e^{\left\langle v_{i} Y_{t}\right\rangle}\right]=\mathrm{e}^{t \Psi_{U}(v)} \tag{25.13}
    \end{equation*}

    then $w$ satisfies (25.12). In fact, $\left\langle v, Y_{t}\right\rangle=\left\langle\left\langle U^{-1}\right\rangle^{\prime} w, Y_{t}\right\rangle=\left\langle w, X_{t}\right\rangle$
    and

    $$
        \begin{aligned}
            \Psi_{U}(v) & =\frac{1}{2}\left\langle U^{\prime} v, A U^{\prime} v\right\rangle+\int\left(\mathrm{e}^{\langle v, U x\rangle}-1-\langle v, U x\rangle 1_{D}(x)\right) v(\mathrm{~d} x)+\langle U \gamma, v\rangle \\
                        & =\Psi(w)
        \end{aligned}
    $$

    by (11.8)-(11.10). That is, (25.13) is identical with (25.12). \\

    Step 3. Given $w \in \mathbb{C}^{d}$ satisfying $\operatorname{Re} w \in C$, we shall
    show (25.12). If Re $w=0$, there is nothing to prove. Assume Re $w \neq 0$. Choose a
    linear transformation $U$ from $\mathbb{R}^{d}$ onto $\mathbb{R}^{d}$ such that $\operatorname{Re} w=U^{\prime} e_{1}$.
    Consider the Lévy process $Y_{t}=U X_{t}$. Since $C_{U}=\left(U^{\prime}\right)^{-1} C$, we have $e_{1} \in C_{U}$.
    We know, by Step 1, that, if $v \in \mathbb{C}^{d}$ satisfies $\operatorname{Re} v=e_{1}$, then (25.13) holds.
    Hence, by the result of Step 2, $w$ satisfies (25.12).

\end{proof}

We close this section with a discussion of the $g$-moments of $\sup _{s \in[0, t]}\left|X_{s}\right|$ \\

THEOREM 25.18. Let $\left\{X_{t}\right\}$ be a Lévy process on $\mathbb{R}^{d}$. Define

\begin{equation*}
    X_{t}^{*}=\sup _{s \in[0, t]}\left|X_{s}\right| \tag{25.14}
\end{equation*}

Let $g(r)$ be a nonnegative continuous submultiplicative function on $[0, \infty)$, increasing to $\infty$ as $r \rightarrow \infty$.
Then the following four statements are equivalent.

(1) $E\left[g\left(X_{t}^{*}\right)\right]<\infty$ for some $t>0$. \\

(2) $E\left[g\left(X_{t}^{*}\right)\right]<\infty$ for every $t>0$. \\

(3) $E\left[g\left(\left|X_{t}\right|\right)\right]<\infty$ for some $t>0$. \\

(4) $E\left[g\left(\left|X_{t}\right|\right)\right]<\infty$ for every $t>0$. \\

\begin{proof}
    Since $g(|x|)$ is submultiplicative on $\mathbb{R}^{d}$, (3) and (4) are equivalent by
    Theorem 25.3. As $\left|X_{t}\right| \leq X_{t}^{*}$, all we have to show is that, for
    any fixed $t>0$, $E\left[g\left(\left|X_{t}\right|\right)\right]<\infty$ implies
    $E\left[g\left(X_{t}^{*}\right)\right]<\infty$. We claim that, for any $a>0$ and $b>0$,
\end{proof}

\begin{equation*}
    P\left[X_{t}^{*}>a+b\right] \leq P\left[\left|X_{t}\right|>a\right] / P\left[X_{t}^{*} \leq b / 2\right] \tag{25.15}
\end{equation*}

Fix $t$ and let $t_{n, j}=j t / 2^{n}$ for $j=1, \ldots, 2^{n}$ and $X_{(n)}^{*}=\max _{1 \leq j \leq 2^{n}}\left|X_{t_{n, j}}\right|$. Choosing $Z_{j}(s)=Z_{j}=X_{t_{n, j}}-X_{t_{n, j-1}}$ in Lemma 20.2 and using Remark 20.3, we have

$$
    P\left[X_{(n)}^{*}>a+b\right] \leq P\left[\left|X_{t}\right|>a\right] / P\left[X_{(n)}^{*} \leq b / 2\right]
$$

in (20.2). Hence, letting $n \rightarrow \infty$, we get (25.15).
Choose $b>0$ such that $P\left[X_{t}^{*} \leq b / 2\right]>0$.
Let $\tilde{g}(r)$ be a continuous increasing function on $[0, \infty)$
                such that $\bar{g}(0)=0$ and $\tilde{g}(r)=g(r)$ for $r \geq 1$.
                Apply Lemma 17.6 to $\left.k(r)=1-P|| X_{t} \mid \leq r\right]$ and $l(r)=\tilde{g}(r)$. Then

$$
    \int_{0+}^{\infty} P\left[\left|X_{t}\right|>r\right] \mathrm{d} \tilde{g}(r)=\int_{(0, \infty)} \tilde{g}(r) P\left[\left|X_{t}\right| \in \mathrm{d} r\right]=E\left[\widetilde{g}\left(\left|X_{t}\right|\right)\right]
$$

14 follows from (25.15) that

$$
    \int_{0+}^{\infty} P\left[X_{t}^{*}>r+b\right] \mathrm{d} \widetilde{g}(r) \leq E\left[\widetilde{g}\left(\left|X_{t}\right|\right)\right] / P\left[X_{t}^{*} \leq b / 2\right]
$$

The integral in the left-hand side equals

$$
    \int_{(0, \infty)} \tilde{g}(r) P\left[X_{t}^{*}-b \in \mathrm{d} r\right]=E\left[\tilde{g}\left(X_{t}^{*}-b\right) ; X_{t}^{*}>b\right]
$$

similarly. Hence, if $E\left[g\left(\left|X_{t}\right|\right)\right]<\infty$,
then $E\left[g\left(X_{t}^{*}-b\right) ; X_{t}^{*}>b\right]<\infty$ and,
by the submultiplicativity of $g, E\left[g\left(X_{t}^{*}\right)\right]<\infty$. \\

Remark 25.19. Let $d=1$. Doob [93], p. 337, shows an explicit bound:

\begin{equation*}
    E\left[\left(X_{t}^{*}\right)^{\alpha}\right] \leq 8 E\left[\left|X_{t}\right|^{\alpha}\right] \text { for } \alpha \geq 1 \tag{25.16}
\end{equation*}


provided that $E\left|X_{t}\right|<\infty$ and $E X_{t}=0$. This is true not only for Lévy processes but also
for additive processes on $\mathbb{R}$. Define the supremum process $M_{t}=$ sup $_{0 \leq \leq \leq \leq t} X_{s}$.
If $\frac{1}{\nu(1, \infty)}[\nu]_{(1, \infty)}$ is subexponential,
then $P\left[M_{t}>x\right] / \nu(x, \infty) \rightarrow t$ as $x \rightarrow \infty$.
This is due to Berman [18] and Rosinski and Samorodnitsky [382]. Extension of this result in the case
where the right tail of $v$ is lighter is studied in Braverman and Samorodnitsky [57] and Braverman [56].


\end{document}