\documentclass[a4paper,11pt]{article}

\setlength{\textwidth}{15.0cm}
\setlength{\textheight}{24.0cm}
\setlength{\topmargin}{0cm}
\setlength{\headsep}{0cm}
\setlength{\headheight}{0cm}
\pagestyle{plain}

\usepackage{amsmath, amsfonts, mathtools, amsthm, amssymb}
\usepackage{import}
\usepackage{pdfpages}
\usepackage{transparent}
\usepackage{xcolor}

\renewcommand{\thesection}{(\arabic{section})}
\setlength{\parindent}{0pt}
\title{Answers: Lévy-Khintchine formula}

\begin{document}

\maketitle
\date{}

\section{}
WTS: $D$ is the unit ball and

\begin{equation} \label{eq: levy measure conditions}
    \nu\big( \{0 \} \big)=0 \ a n d \ \int_{\mathbb{R}^{d}} ( | x |^{2} \wedge1 \big) \nu\big( d x \big) < \infty,
\end{equation}
set
\begin{equation}
    f(t,x) =  ( \operatorname{e x p} ( i \langle t, x \rangle)-1-i \langle t, x \rangle I_{D} ( x ) )
    .
\end{equation}

then following function is continuous:

\begin{equation}
    t \mapsto\int_{\mathbb{R}^{d}} f(t,x)\nu( d x )
    .
\end{equation}

\begin{proof}
    Take $t_{k} \rightarrow t \implies \sup_{k>0}(t_{k})<\infty, \sup_{k>0}(|t_{k}|^{2})<\infty $
    (if the supremum is $\infty$ there exist a subsequence that converges to $\infty$ which is a contradiction) and

    \begin{align}
        |f(t_{k},x)| & \le \frac{1}{2} |t_{k}|^{2} |x|^{2} I_{D}(x) + 2 I_{D^{c}}(x)                                              \\
                     & \le \frac{1}{2} \sup_{k}\left(|t_{k}|^{2}  \right)  |x|^{2} I_{D}(x) + 2 I_{D^{c}}(x) \label{eq:dominator}
        .
    \end{align}
    so $f(t_{k},x)$ is dominated (\ref{eq:dominator}) which is $\nu$-integrable because of (\ref{eq: levy measure conditions}).
    Now by the dominated convergence theorem we have

    \begin{align}
         & \lim_{k \rightarrow \infty} {\int_{\mathbb{R}^{d}} f(t_{k},x)\nu( d x )}   \\
         & =  \int_{\mathbb{R}^{d}} \lim_{k \rightarrow \infty}{f(t_{k},x)\nu( d x )} \\
         & =  \int_{\mathbb{R}^{d}} {f(t,x)\nu( d x )}
        .
    \end{align}

\end{proof}

\section{}
WTS: compound Poisson process has a Lévy-Khintchine representation.
\begin{proof}
    To see this rewrite the characteristic function of a compound Poisson process $X$:
    \begin{align}
        \phi_{X}(t) & = \exp \left(  \lambda\int_{\mathbb{R}^{d}} ( \exp ( i \langle t, x \rangle)) Q ( d x )-1  \right)                                                                       \\
                    & = \exp \left(  \lambda\int_{\mathbb{R}^{d}} ( \exp ( i \langle t, x \rangle)-1) Q ( d x )  \right)                                                                       \\
                    & = \exp \left(  \int_{\mathbb{R}^{d}} ( \exp ( i \langle t, x \rangle)-1) \lambda Q ( d x )  \right)                                                                      \\
                    & = \exp \left(  \int_{\mathbb{R}^{d}} ( \exp ( i \langle t, x \rangle)-1- i \langle t, x \rangle I_{D}(x) + i \langle t, x \rangle I_{D}(x) )  \lambda Q ( d x )  \right) \\
                    & = \exp \left(  \int_{\mathbb{R}^{d}} ( \exp ( i \langle t, x \rangle)-1- i \langle t, x \rangle  I_{D}(x)) \lambda Q ( d x )
        +  i \langle t, \int_{D}x\lambda Q ( d x ) \rangle      \right)
    \end{align}
\end{proof}

\section{}
Question: What can you say about the infinitely divisible distribution if $\nu$ the Lévy measure is finite? \\
Answer: The infinitely divisible distribution is the sum of a independent normal distribution and a compound Poisson distribution.
\begin{align}
     & \exp \left[-\frac{1} {2} \langle t, A t \rangle+i \langle\gamma, t \rangle+\int_{\mathbb{R}^{d}} \left( \exp ( i \langle t, x \rangle)-1-i \langle t, x \rangle I_{D} ( x ) \right) \nu( d x ) \right] \\
     & =\exp \left[-\frac{1} {2} \langle t, A t \rangle+i \langle\gamma, t \rangle+\int_{\mathbb{R}^{d}} \left( \exp ( i \langle t, x \rangle)-1 \right)\nu( d x )
        -\int_{\mathbb{R}^{d}}
    \left(i \langle t, x \rangle I_{D} ( x ) \right) \nu( d x ) \right]                                                                                                                                       \\
     & =\exp \left[-\frac{1} {2} \langle t, A t \rangle+i \langle\gamma -\int_{D}x \nu(dx) , t \rangle \right]
    \exp \left[ \int_{\mathbb{R}^{d}} \left( \exp ( i \langle t, x \rangle)-1 \right)\nu( d x ) \right]                                                                                                       \\
     & =\exp \left[-\frac{1} {2} \langle t, A t \rangle+i \langle\gamma -\int_{D}x \nu(dx) , t \rangle \right]
    \exp \left[ \int_{\mathbb{R}^{d}} \left( \exp ( i \langle t, x \rangle)-1 \right)\nu( \mathbb{R}^{d}) \frac{\nu( d x  )}{\nu(\mathbb{R}^{d})} \right]
    .
\end{align}
The terms of the product are easily recognized as the characteristic
functions of a normal distribution and a compound Poisson distribution.

\section{}
WTS:
\begin{equation}
    s_{k}>0\rightarrow \infty \implies
    - \frac{1} {2} \langle t, A t \rangle+i \langle\gamma, t \rangle s_k^{-1}+\int_{\mathbb{R}^{d}} \frac{\exp ( i s_k \langle t, x \rangle)-1-i s_k \langle t, x \rangle I_{D} ( x )} {s_k^{2}} \nu( d x )
    \rightarrow  \frac{- \langle t, A t \rangle}{2}
    .
\end{equation}

\begin{proof}
    Assume $s_{k}\rightarrow \infty$ set
    \begin{equation}
        f(t,x) =  ( \exp ( i \langle t, x \rangle)-1-i \langle t, x \rangle I_{D} ( x ) ),
    \end{equation}
    then all we have to show is that (the second term goes to zero)
    \begin{equation}
        \int_{\mathbb{R}^{d}} f(s_{k}t,x) \nu( d x ) \rightarrow 0
        .
    \end{equation}
    Use the same dominator (\ref{eq:dominator}):
    \begin{align}
        \left|\frac{f(s_{k}t,x)}{s_{k}^{2}}\right| & \le \frac{\frac{1}{2} |s_{k} t|^{2} |x|^{2} I_{D}(x) + 2 I_{D^{c}}(x)}{ s_{k}^{2}}  \\
                                                   & \le \frac{1}{2} |t|^{2} |x|^{2} I_{D}(x) + \frac{2I_{D^{c}(x)}}{\inf_{k} s_{k}^{2}}
        .
    \end{align}
    If $\inf_{k} s_{k}^{2}= 0$ then because $s_k^{2}>0$ there exist a subsequence that
    converges to $0$ which is a contradiction with $s_{k}>0 \rightarrow \infty$.
    Again we can apply the dominated convergence theorem:

    \begin{align}
         & \lim_{k \to \infty} \left|\int_{\mathbb{R}^{d}} \frac{\exp ( i s_k \langle t, x \rangle)-1-i s_k \langle t, x \rangle I_{D} ( x )} {s_k^{2}} \nu( d x ) \right|                  \\
         & \le\lim_{k \to \infty} \int_{\mathbb{R}^{d}} \left|\frac{\exp ( i s_k \langle t, x \rangle)-1-i s_k \langle t, x \rangle I_{D} ( x )} {s_k^{2}}\right|  \nu( d x )               \\
         & \le \int_{\mathbb{R}^{d}}\lim_{k \to \infty} \left|\frac{\exp ( i s_k \langle t, x \rangle)-1-i s_k \langle t, x \rangle I_{D} ( x )} {s_k^{2}}\right|  \nu( d x )               \\
         & \le \int_{\mathbb{R}^{d}}\lim_{k \to \infty} \frac{\left|\exp ( i s_k \langle t, x \rangle)-1-i s_k \langle t, x \rangle I_{D} ( x )\right|} {s_k^{2}}  \nu( d x )               \\
         & \le \int_{\mathbb{R}^{d}}\lim_{k \to \infty} \frac{\left|\exp ( i s_k \langle t, x \rangle)-1\right|+ \left|i s_k \langle t, x \rangle I_{D} ( x )\right|} {s_k^{2}}  \nu( d x ) \\
         & \le \int_{\mathbb{R}^{d}}\lim_{k \to \infty} \frac{2+ \left|i s_k \langle t, x \rangle I_{D} ( x )\right|} {s_k^{2}}  \nu( d x )                                                 \\
         & \rightarrow 0
    \end{align}
\end{proof}

\section{}
WTS:
Set:
\begin{equation}
    f(t,x) =  ( \exp ( i \langle t, x \rangle)-1-i \langle t, x \rangle I_{D} ( x ) ),
\end{equation}
then
\begin{equation}
    \int_{|x| \ge \frac{1}{n} } f(s_{k}t,x) \nu( d x )\rightarrow_{n} \int_{\mathbb{R}^{d}} f(s_{k}t,x) \nu( d x )
    .
\end{equation}
\begin{proof}
    Use the same dominator (\ref{eq:dominator}) and apply dominated convergence theorem.
\end{proof}

\section{}
WTS: $\int_{[-h,h]^{d}} \psi_{n}(t) dt  \in  \mathbb{R}$ without calculating it.
\begin{proof}
    Didn't manage to prove it without calculating it. But here is an idea: we know
    because of the Lévy-Khintchine formula (which we can't get without the calculation) it is
    easily seen that $\psi(t)$ is conjugate to $\psi(-t)$,  $|\psi|=1 $ and we are integrating/ summing over
    a even domain ($t \in D \implies -t \in D$) so the complex parts cancel out.
\end{proof}

\section{}
Question: Why can we apply Fubini's theorem in line 1 page 23. \\
Answer: Because the conditions are satisfied this is already well explained in the notes.

\section{}
Question: Why is line 15 on page 13 easy to see? \\
Answer: It is a product of multiple sinc functions which are individually bounded between $-1$ and $1$ 
and reach their maximum at $0$ which is $1$. So the product is continuous also bounded between $-1$ and $1$
and only reaches $1$ in $0$. 

\section{}


\end{document}