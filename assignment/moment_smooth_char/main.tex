\documentclass[a4paper,12pt]{article}

\setlength{\textwidth}{15.0cm}
\setlength{\textheight}{24.0cm}
\setlength{\topmargin}{0cm}
\setlength{\headsep}{0cm}
\setlength{\headheight}{0cm}
\pagestyle{plain}

\usepackage{amsmath, amsfonts, mathtools, amsthm, amssymb}
\usepackage{import}
\usepackage{pdfpages}
\usepackage{transparent}
\usepackage{xcolor}

\input{environments.tex}

\setlength{\parindent}{0pt}

\begin{document}

\begin{definition}[$\Delta^{k}_{h}$]
    We define $\forall h>0: \Delta_{h}$ ($h$-central difference) of $f:\mathbb{R}\rightarrow \mathbb{R}$ as follows:
    \begin{equation}
        \Delta_{h} f(t) = f(t+h)- f(t-h),
    \end{equation}
    and $\forall k \in \mathbb{N}:\Delta^{k}_{h}$ is defined as applying $\Delta_{h}$ applying $k$ times and
    $k=0$ as the identity. Note that $\Delta^{k}_{h}$ is a linear operator. We will always take
    differences in respect to the $t$ variable.
\end{definition}

\begin{lemma}[$\Delta^{k}_{h} e^{at}$]
    \begin{equation}
        \forall k \in \mathbb{N}: \Delta^{k}_{h} e^{at} =  e^{at} \left( e^{ah} - e^{-ah} \right)^{k}.
    \end{equation}
\end{lemma}

\begin{proof}
    This follows simply by induction. $k=0$ is trivial.
    Base case:
    \begin{align}
        \Delta_{h} e^{at} & =\left( e^{a(t+h)} - e^{a(t-h)} \right) \\
                          & =e^{at}\left( e^{ah} - e^{-ah} \right)
        .
    \end{align}
    Induction hypothesis:
    \begin{equation}
        \Delta^{k-1}_{h} e^{at} = e^{at} \left( e^{ah} - e^{-ah} \right)^{k-1}.
    \end{equation}

    Induction step:
    \begin{align}
        \Delta^{k}_{h} e^{at} & = \Delta_{h} \Delta^{k-1}_{h} (e^{at})                                        \\
                              & = \Delta_{h} \left( e^{at} \left( e^{ah} - e^{-ah} \right)^{k-1} \right)      \\
                              & = \Delta_{h} \left( e^{at}\right) \left( e^{ah} - e^{-ah} \right)^{k-1}       \\
                              & = e^{at}\left( e^{ah} - e^{-ah} \right) \left( e^{ah} - e^{-ah} \right)^{k-1} \\
                              & =  e^{at} \left( e^{ah} - e^{-ah} \right)^{k}.
    \end{align}
\end{proof}

% \begin{lemma}[$\Delta c$]
%     \begin{equation}
%         \forall c \in \mathbb{R}: \Delta_{h} c =c-c = 0
%         .
%     \end{equation}
% \end{lemma}


% \begin{lemma}[$\Delta_{h} t^{n}$]
%     $\forall n>0 \in \mathbb{N}: \Delta_{h} t^{n}$ is a polynomial of degree $n-1$
%     with leading coefficient $2hn$.
% \end{lemma}

% \begin{proof}
%     \begin{equation}
%         \Delta_{h} t^{n} = (t+h) ^{n}- (t-h) ^{n}  .
%     \end{equation}
%     So $\Delta_{h} t^{n}$  is a polynomial, verify the degree and leading coefficient by looking at its derivatives
%     (or use the binomial theorem):
%     \begin{align}
%         \frac{d^{n}}{dt^{n}}\left((t+h) ^{n}- (t-h) ^{n} \right)     & = n! -n! =0.         \\
%         \frac{d^{n-1}}{dt^{n-1}}\left((t+h) ^{n}- (t-h) ^{n} \right) & = n!(t+h) - n!(t-h). \\
%                                                                      & = 2h n! \neq 0.
%     \end{align}

% \end{proof}

% \begin{lemma}[$\Delta^{k}_{h} t^{n}$]
%     $\forall n,k \in \mathbb{N}, n<k: \Delta^{k}_{h} t^{n} = 0$.
% \end{lemma}

% \begin{proof}
%     Follows from the previous $2$ lemmas and linearity of $\Delta^{k}_{h}$.
% \end{proof}

% \begin{lemma}[$\Delta^{k}_{h} t^{k}$]
%     $\forall k \in \mathbb{N}: \Delta^{k}_{h} t^{k} = (2h)^{k} k!$.
% \end{lemma}

% \begin{proof}
%     Proof by induction. Base case $k=0$ is trivial. Induction hypothesis:
%     \begin{equation}
%         \forall l<k:  \Delta^{l}_{h} t^{l} = (2h)^{l} l!.
%     \end{equation}
%     Induction step, use previous lemmas:
%     \begin{align}
%         \Delta^{k}_{h} t^{k} & = \Delta^{k-1}_{h}\Delta_{h} t^{k}                     \\
%                              & = \Delta^{k-1}_{h}\left( (t+h)^{k} - (t-h)^{k} \right) \\
%                              & = \Delta^{k-1}_{h}\left( 2hk t^{k-1} + ... \right)     \\
%                              & = \Delta^{k-1}_{h}\left( 2hk t^{k-1}\right)            \\
%                              & =2hk  \Delta^{k-1}_{h}\left( t^{k-1}\right)            \\
%                              & =2hk  (2h)^{k-1} (k-1)!                                \\
%                              & =  (2h)^{k} k!
%     \end{align}
% \end{proof}


\begin{theorem}[central finite difference formula]
    $\forall k \in \mathbb{N}$ $\forall$  $k$-continuously differentiable $f$'s at $0$:
    $\exists \delta>0: \forall |s|<\delta: $
    \begin{equation}
        f^{(k)}(s) = \lim_{h \to 0} \frac{\Delta^{k}_{h} f(t)|_{t=s}}{(2h)^{k}}.
    \end{equation}
\end{theorem}

\begin{proof}
    Follows by induction and the mean value theorem. Base case $k=0$ is trivial.
    Induction hypothesis:
    \begin{equation}
        \exists \delta>0: \forall |s|<\delta: \forall l<k: f^{(l)}(s) = \lim_{h \to 0} \frac{\Delta^{l}_{h} f(t)|_{t=s}}{(2h)^{l}}
        .
    \end{equation}
    Induction step:
    \begin{align}
         & \lim_{h \to 0} \frac{\Delta^{k}_{h} f(t)|_{t=s}}{(2h)^{k}}                                         \\
         & = \lim_{h \to 0} \frac{\Delta_{h}\Delta^{k-1}_{h} f(t)|_{t=s}}{(2h)^{k}}                           \\
         & = \lim_{h \to 0} \frac{\Delta^{k-1}_{h} f(t)(|_{t=s+h}-|_{t=s-h})}{(2h)^{k}}                       \\
         & = \lim_{h \to 0} \frac{2h (\Delta^{k-1}_{h} f(t))'|_{t=z_{h}}}{(2h)^{k}}, z_{h} \in[s-h,s+h]       \\
         & = \lim_{h \to 0} \frac{\Delta^{k-1}_{h} f'(t)|_{t=z_{h}}}{(2h)^{k-1}} = f'^{(k-1)}(s) = f^{(k)}(s)
        .
    \end{align}
    The induction hypothesis may be applied because $\Delta^{k-1}_{h} f'(t)$ is $k-1$-continuously differentiable.

    % Follows from Taylor's theorem and previous lemma.
    % \begin{align}
    %      & \lim_{h \to 0} \frac{\Delta^{k}_{h}|_{t=0} f(t)}{(2h)^{k}}                                                                   \\
    %      & = \lim_{h \to 0} \frac{\Delta^{k}_{h}|_{t=0} \left( \sum_{n=0}^{k-1} \frac{f^{(n)}(0)}{n!} t^n
    %     + \left(\frac{f^{(k)}(0)}{k!} + h_{k}(t)\right)t^{k} \right)  }{(2h)^{k}}                                                       \\
    %      & = \lim_{h \to 0} \frac{\Delta^{k}_{h}|_{t=0} \left( \left(\frac{f^{(k)}(0)}{k!} + h_{k}(t)\right)t^{k} \right)  }{(2h)^{k}}  \\
    %      & = \lim_{h \to 0} \frac{\left(\frac{f^{(k)}(0)}{k!} + h_{k}(t)\right) \Delta^{k}_{h}|_{t=0} \left( t^{k} \right)  }{(2h)^{k}} \\
    %      & = \lim_{h \to 0} \frac{\left(\frac{f^{(k)}(0)}{k!} + h_{k}(t)\right) (2h)^{k} k!  }{(2h)^{k}}                                \\
    %      & = \lim_{h \to 0} f^{(k)}(0) + h_{k}(t) k!                                                                                    \\
    %      & = f^{(k)}(0)
    % \end{align}
\end{proof}


\begin{lemma}
    $\forall k \in  \mathbb{N}:\phi_{X}(t)$ is $2k$-time continuous differentiable $\Rightarrow  E[X^{2k}] < \infty$
\end{lemma}

\begin{proof}
    Use the central finite difference formula and previous lemma's:
    \begin{align}
        |\phi_{X}^{(2k)}(0)| & = \left|\lim_{h \to 0} \frac{\Delta^{2k}_{h}|_{t=0} \phi_{X}(t)}{(2h)^{2k}}\right|                      \\
                             & = \left|\lim_{h \to 0} \frac{\Delta^{2k}_{h}|_{t=0} E\left[e^{itX}\right]}{(2h)^{2k}}\right|            \\
                             & = \left|\lim_{h \to 0} \frac{ E\left[\Delta^{2k}_{h}|_{t=0}e^{itX}\right]}{(2h)^{2k}}\right|            \\
                             & = \left|\lim_{h \to 0} \frac{ E\left[e^{itX} |_{t=0} (e^{ihX}-e^{-ihX})^{2k}\right]}{(2h)^{2k}}\right|  \\
                             & = \left|\lim_{h \to 0} \frac{ E\left[(e^{ihX}-e^{-ihX})^{2k}\right]}{(2h)^{2k}}\right|                  \\
                             & = \left|\lim_{h \to 0} \frac{1}{i^{2k}}E\left[\frac{ (e^{ihX}-e^{-ihX})^{2k}}{(2h)^{2k}} \right]\right| \\
                             & = \lim_{h \to 0} E\left[\frac{ \sin(Xh)^{2k}}{h^{2k}}\right]                                            \\
                             & = \lim_{h \to 0} E\left[ X^{2k}\frac{ \sin(Xh)^{2k}}{X^{2k} h^{2k}}\right]                              \\
                             & = \lim_{h \to 0} E\left[ X^{2k} \operatorname{sinc}(Xh) ^{2k}\right]                                    \\
                             & \ge  E\left[ X^{2k} \right]
    \end{align}
    Where last line follows by Fatou's lemma.
\end{proof}


\end{document}